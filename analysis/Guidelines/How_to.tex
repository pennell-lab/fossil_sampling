
% ---------------------------------------------------------------------------- %
  \documentclass[12pt,twoside,a4paper,pdftex]{scrbook}

% ---------------------------------------------------------------------------- %
  % Pacotes 
% ,top=2.54cm,bottom=2.0cm,left=2.0cm,right=2.54cm % old definition of the margins....
\usepackage[a4paper,top=2.54cm,bottom=2.0cm,left=2.0cm,right=2.54cm]{geometry} 
\usepackage[utf8x]{inputenc}
\usepackage[T1]{fontenc}
\usepackage{textcomp} 
\usepackage{setspace} 
\usepackage{graphicx} 
\usepackage{color}
\usepackage{nameref}
\usepackage{hyperref}
\usepackage{layout} 
\usepackage{rotating}
\usepackage{amsthm}
\usepackage{amsmath}

\usepackage{bibentry}
\nobibliography*


\usepackage{natbib} % comando para poder citar authordate. permite o uso de citep, citet, etc


\definecolor{graytab}{gray}{0.85}


\begin{document}
\section{Examples of fossil record simulations}
\subsection{From \citet{Silvestro14}}

Simulated data sets were generated based on six patterns of species diversification chosen to yield scenarios of varying species richness commonly observed in empirical data (Fig. 2 and Table 1; Sepkoski 1981):

\begin{itemize}
\item I. expanding diversity with constant speciation and extinction rates ($ \lambda > \mu $);
\item II. expanding diversity followed by a decline with all taxa going extinct before the present ($ \lambda_1 > \mu_1, \lambda_2 < \mu_2 $);
\item III. expanding and then declining diversity followed by turnover at equilibrium ($ \lambda_1 > \mu_1, \lambda_2 < \mu_2, \lambda_3 = \mu_3 $);
\item IV. expanding diversity followed by turnover at equilibrium due to a decrease in speciation rate ($ \lambda_1 > \mu_1, \lambda_2 = \mu_2 $);
\item V. expanding diversity followed by turnover at equilibrium due to a decrease in speciation rate and increase in extinction rate ($ \lambda_1 > \mu_1, \lambda_2 = \mu_2 $); and
\item VI. constant speciation rate and a mass extinction event ($ \lambda_1 > \mu_1, \lambda_2 << \mu_2, \lambda_1 > \mu_1 $).
\end{itemize}

Based on the complete birth–death realizations, fossil occurrences were simulated for each species. Along each lineage i a number of occurrences Ki was derived from a Poisson distribution with rate parameter qPOI = q(si − ei ) where q is the preservation rate (cf. Equation 1), and si,ei are the true times of speciation and extinction. The preservation rate was assumed q = 3 unless stated otherwise. A number of fossils Ki were then randomly drawn from the PERT distribution for all species, resulting in a synthetic data set that mimics the fossil record. The shape parameter of the PERT distribution was assumed to be l = 4 (as in Equation 3) unless stated otherwise. The number of occurrences K depended only on the preservation rate q and on the species lifespan, i.e., it was not conditioned on being greater than 0. The lineages without a fossil record (K = 0) were disregarded in the analyses because of the condition stated in Equation (4). All extant species were then truncated at time 0 (i.e., the present). The number of extant species based on the complete birth–death realization (NOBS) was kept to construct the hyperprior of Equations (12–15).


\subsection{From \cite{Foote00}}

Preservation can be modeled in a number of realistic ways that include variation in time and space (Shaw 1964; Koch and Morgan 1988; Marshall 1994; Holland 1995; Holland and Patzkowsky 1999; Weiss and Marshall 1999). As a heuristic tool for understanding the behavior of diversity and rate measures, it is convenient to focus on the temporal aspect and to start by assuming time-homogeneous fossil preservation at a constant per-capita rate r per Lmy (Paul 1982, 1988; Pease 1985; Strauss and Sadler 1989; Marshall 1990; Foote and Raup 1996; Solow and Smith 1997; Foote 1997). This simple assumption will be relaxed below. In the time-homogeneous case, the proportion of lineages preserved is equal to r/(q 1 r) if p 5 q and if the fossil record is of effectively infinite length (Pease 1985; Solow and Smith 1997) (see Edge Effects, below). It is therefore natural for many problems to express preservation rate as a multiple of q. Throughout this discussion I will assume taxonomic homogeneity of taxonomic and pres- ervational rates. For modeling, this assumption can easily be relaxed by performing calculations for an arbitrary number of rate clas- ses and combining the results (see Buzas et al. 1982, Koch and Morgan 1988, Holland 1995, Holland and Patzkowsky 1999, and Weiss and Marshall 1999 for explicit treatments of taxo- nomic heterogeneity of preservation)



\section{Other studies comparing methods in the FR}

See:

\begin{itemize} 
\item \bibentry{Alroy00}
\item \bibentry{Alroy10}
\item \bibentry{Alroy14}
\end{itemize}




% ---------------------------------------------------------------------------- %
% Bibliografia
\singlespacing 
\bibliographystyle{plainnat}
\bibliography{refs}



\end{document}











